\paragraph{Treść}~\\
Zaimplementować program, w którym zmienna jest inkrementowana lub dekrementowana przez dwa wątki.

\lstinputlisting[caption=Kod źródłowy klasy Main do zadania,label=lst:MainJava]{Main.java}

\lstinputlisting[caption=Skrypt do generowania wyników do zadania,label=lst:ZadanieSh]{zadanie.sh}

\lstinputlisting[caption=Wyniki działania programu do zadania,label=lst:ZadanieDat]{zadanie.dat}

\paragraph{Podsumowanie}~\\
Program widoczny na listingu~\ref{lst:MainJava} pokazuje problem synchronizacji dostępu do zasobu przez wiele wątków programu.
Jego wyniki zostały wygenerowane za pomocą skryptu widocznego na listowaniu~\ref{lst:ZadanieSh} i są widoczne na listowaniu~\ref{lst:ZadanieDat}.
Można zauważyć, że mimo że program nie ma ma w sobie logiki która miałaby wskazywać na losowość wyników, to rezultat jest inny.
Takie zachowanie programu nazywamy niedeterminizem i jest ono niepożądane w większości przypadków.
Dlatego stosuje się różne sposoby i algorytmy do synchronizacji dostępu do zasobu procesów lub wątków programie.
