\paragraph{Treść}~\\
Proszę zaimplementować semafor z użyciem mechanizmu monitora w Javie (szkielet programu).
Zastosować semafor do problemu wyścigu.

\lstinputlisting[caption=Szkielet programu załączony do zadania,label=lst:SemaforMonJava]{SemaforMon.java}

\lstinputlisting[caption=Kod klasy Semafor do zadania,label=lst:SemaforJava]{Semafor.java}

\lstinputlisting[caption=Przerobiony kod źródłowy klasy Main do zadania z poprzedniego labolatorium,label=lst:MainJava]{Main.java}

\lstinputlisting[caption=Skrypt do generowania wyników do zadania,label=lst:ZadanieSh]{zadanie.sh}

\lstinputlisting[caption=Wyniki działania programu do zadania,label=lst:ZadanieDat]{zadanie.dat}

\paragraph{Podsumowanie}~\\
Semafor w rozwiązaniu widoczny jest na listowaniu~\ref{lst:SemaforJava} i został zaimplementowany z pomocą szkeletu programu załączonego do zadania widocznego na listowaniu~\ref{lst:SemaforMonJava}.
Używa on metod wait() i notify() w bloku synchronized ustawionym na instancję klasy Semafor.
Dzięki temu wątki, które operują na tym samym semaforze mają do niego dosęp synchroniczny.
Pozwala to na poprawienie problemu wyścigu, widocznego na listowaniu~\ref{lst:MainJava}, z poprzedniego zadania.
Wyniki poprawionego programu widoczne na listowaniu~\ref{lst:ZadanieDat} zostały wygenerowane za pomocą skryptu z listowania~\ref{lst:ZadanieSh}.
Widać na nich, że program działa prawidłowo przy wszytkich uruchomieniach.
