\paragraph{Treść}~\\
Problem czytelników i pisarzy proszę rozwiązać przy pomocy:\\
- Semaforów\\
- Zmiennych warunkowych\\
Rozważyć przypadki z faworyzowaniem czytelników, pisarzy oraz z użyciem kolejki FIFO.\\
Proszę wykonać pomiary dla każdego rozwiązania dla różnej ilości czytelników (10 - 100) i pisarzy (od 1 do 10).\\
W sprawozdaniu proszę narysować wykres czasu w zależności od ilości wątków i go zinterpretować.

\lstinputlisting[caption=Kod źródłowy klasy Main do zadania,label=lst:MainJava]{Main.java}

\lstinputlisting[caption=Kod źródłowy klasy ReadersWritersProblem do zadania,label=lst:ReadersWritersProblemJava]{ReadersWritersProblem.java}

\lstinputlisting[caption=Kod źródłowy klasy ReadersWritersProblemSemaphores do zadania,label=lst:ReadersWritersProblemSemaphoresJava]{ReadersWritersProblemSemaphores.java}

\lstinputlisting[caption=Kod źródłowy klasy ReadersWritersProblemConditions do zadania,label=lst:ReadersWritersProblemConditionsJava]{ReadersWritersProblemConditions.java}

\lstinputlisting[caption=Skrypt do generowania danych do zadania,label=lst:ZadanieSh]{zadanie.sh}

\lstinputlisting[caption=Komendy dla programu gnuplot do wygenerowania wykresów do zadania,label=lst:ZadanieGpi]{zadanie.gpi}

\begin{figure}[p]
  \caption{Wynik programu gnuplot dla rozwiązania z semaforami na podstawie komend do zadania}
  \label{fig:SemaphoresJpg}
  \centering
  \includegraphics[width=\textwidth]{semaphores.jpg}
\end{figure}

\begin{figure}[p]
  \caption{Wynik programu gnuplot dla rozwiązania ze zmiannymi warunkowymi na podstawie komend do zadania}
  \label{fig:ConditionsJpg}
  \centering
  \includegraphics[width=\textwidth]{conditions.jpg}
\end{figure}

\paragraph{Podsumowanie}~\\
Problem czytalników i pisarzy został rozwiązany przy pomocy algorytmów opisanych pseudokodem z użyciem semaforów~\cite{WikipediaReadersWritersProblem} oraz zmiennych warunkowych~\cite{GithubAngraveSystemProgrammingSynchronization7}.
Problem jest abstrakcyjnie zaimplementowany w klasie ReadersWritersProblem widocznej na listowaniu ~\ref{lst:ReadersWritersProblemConditionsJava}.
Implementacja wspomnianych wcześniejszych algorytmów do synchronizacji, jest widoczna w klasach odpowiednio ReadersWritersProblemSemaphores na listowaniu ~\ref{lst:ReadersWritersProblemSemaphoresJava} oraz ReadersWritersProblemConditions na listowaniu~\ref{lst:ReadersWritersProblemConditionsJava}.
Wyniki programu są widoczne osobno dla rozwiązania z użyciem semaforów na wykresie~\ref{fig:SemaphoresJpg}, a z użyciem zmiennych warunkowych na wykresie~\ref{fig:ConditionsJpg}.
Zostały one wygenerowane za pomocą programu Gnuplot przez przekazanie komend widocznych na listowaniu~\ref{lst:ZadanieGpi}.
Do wygenerowania serii danych dla wykresów posłużył skrypt widoczny na listowaniu~\ref{lst:ZadanieSh}.
Dla przejrzystości wykresów dane zostały ręcznie przefiltrowane w celu wyszukania błędnych pomiarów, które mogły pojawić się ze względu na wielozadaniowość maszyny na której były one generowane.
Z wykresów można odczytać, że największy wpływ na czas wykonywania programu ma ilość czytelników.
Ilość pisarzy w podanym w treści zadania przedziale nie ma tak dużego wływu na czas wykonywania programu.
Oba algorytmy radzą sobie z problemem synchronizacji w podobny czasowo sposób.
Różnice widoczne są w priorytetyzowaniu czytelników i pisarzy.
Logi pozostawiane przez programy wskazywały, że algorytm z użyciem semaforów równo rozdzielał dostęp pomięczy czytelników, a pisarzy.
Można było zauważyć, że dostęp ten był proporcjonalny do ustawionej ilości pisarzy i czytelników.
Algorytm z użyciem zmiannych warunkowych faworyzował jednak pisarzy.
Z logów programu często można było wyczytać, że pisarze częściej dostawali się do zasobu i szybciej zakańczali swoje działanie.

\begin{thebibliography}{9}
  \bibitem{WikipediaReadersWritersProblem} Readers–writers problem - Wikipedia https://en.wikipedia.org/wiki/Readers\%E2\%80\%93writers\_problem\#Third\_readers-writers\_problem\\
  \bibitem{GithubAngraveSystemProgrammingSynchronization7} Synchronization, Part 7: The Reader Writer Problem https://github.com/angrave/SystemProgramming/wiki/Synchronization\%2C-Part-7\%3A-The-Reader-Writer-Problem\#attempt-4
\end{thebibliography}
